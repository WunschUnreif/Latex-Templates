\documentclass[lineno=off]{wunschnote}
\usepackage{lipsum}

\noteconfig{
    title = How to use \code{wunschnote} to write notes,
    course = wunschnote,
    lecture = 1,
}

\begin{document}
    \maketitle

    \section{Titles}
    \subsection{subsection}
    \subsubsection{subsubsection}
    \paragraph{paragraph} These are title styles

    \paragraph{Another paragraph}
    \lipsum[2]

    \subsection{Theorems}

    \begin{theorem}[named theorem]
        This is a theorem with name.
    \end{theorem}

    \begin{theorem}
        No name theorem.
    \end{theorem}

    \begin{lemma}
        This is a lemma.
    \end{lemma}

    \begin{corollary}
        This is a corollary.
    \end{corollary}

    \begin{definition}
        This is a definition.
    \end{definition}

    \begin{example}
        This is an example.
    \end{example}

    \begin{remark}
        This is a remark.
    \end{remark}

    \begin{remark*}
        Each theorem-like environment has a starred version which doesn't increase nor show the counter.
    \end{remark*}

    \begin{proof}
        This document class is a trivial result.
    \end{proof}

    \section{Boxes}
    \label{sec:boxes}
    \begin{wunschbox}[A fantastic box]
        You can use the box with environment \verb|deanbox|.
    \end{wunschbox}

    \begin{wunschbox}[]
        A \verb|deanbox| takes 2 optional arguments. The first one is the title of the box. Of course you can get a box without title.
    \end{wunschbox}

    \begin{wunschbox}[Setting box colors][\successcolor]
        The second optional argument controls the color of the box. You should only pass the color series name such as \verb|green| or use the theme color such as \verb|\successcolorname|. By default, the box frame and the title bar use the base color and the box background uses the \verb|-lighten-5| variation.
        \par 
        Paragraph break inside boxes.
    \end{wunschbox}

    There is also an inline version of boxes using command \code{\textbackslash inlbox}. Since inline boxes are preferable to display short codes, we also provide the command \code{\textbackslash code} for such purposes, which is equivalent to a \code{\textbackslash texttt} inside a \code{\textbackslash inlbox}.

    \section{Lorem Ipsum}
    \lipsum

    \section{Enum Item}
    An \code{enumerate}:
    \begin{enumerate}
        \item enum 1
        \item enum 2
        \item enum paragraph \lipsum[2]
    \end{enumerate}
    Large numbers:
    \begin{enumerate}[start=9]
        \item enum 1
        \item enum 2
        \item enum paragraph \lipsum[2]
    \end{enumerate}

    A nested \code{enumerate}:
    \begin{enumerate}
        \item First item.
        \item Nested 
        \begin{enumerate}
            \item Item 2.1
            \item Item 2.2
            \item Level 3:
            \begin{enumerate}
                \item Item 2.3.1
                \item Item 2.3.2
            \end{enumerate}
        \end{enumerate}
    \end{enumerate}

    An \code{description}:
    \begin{description}
        \item[First] first item.
        \item[Second] second item.
    \end{description}

    An \code{itemize}:
    \begin{itemize}
        \item First item.
        \item Nested 
        \begin{itemize}
            \item Item 2.1
            \item Item 2.2
            \item Level 3:
            \begin{itemize}
                \item Item 2.3.1
                \item Item 2.3.2
            \end{itemize}
        \end{itemize}
    \end{itemize}
\end{document}
