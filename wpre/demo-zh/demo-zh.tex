\documentclass[no-math, aspectratio=1610, 10pt]{beamer}
\usetheme[font=harmony-os]{wpre}

\usepackage{amsmath}
\usepackage{amsfonts}
\usepackage{amssymb}
\usepackage{mathabx}
\usepackage{verbatim}
\usepackage{expl3}
\usepackage{lipsum}


\title{Beamer演示文稿主题 \texttt{wpre}}
\subtitle{快速上手教程}
\series{Hello, \texttt{wpre}!}
\credits{感谢IPADS实验室的课程幻灯片为本模板带来的启示!感谢HarmonyOS Sans字体!}
\author{WunschUnreif}
\institute{Basics, SJTU}
\date{\today}

\begin{document}
    \maketitle

    \begin{frame}{目录}
        \tableofcontents
    \end{frame}

    \section{主题选项}

    \begin{frame}[fragile]
        \frametitle{为\texttt{wpre}指定选项}
        可以在导言区加载本主题时指定选项:
        \begin{verbatim}
    \usetheme[key = val, ...]{wpre}
        \end{verbatim}
        也可以在文档中任意位置修改选项: 
        \begin{verbatim}
    \wpreset{key = val, ...}
        \end{verbatim}
    \end{frame}

    \begin{frame}
        \frametitle{可用的选项}
    
        元数据:
        \begin{description}
            \item[\texttt{series}] 演示文稿所属的系列,例如课程代码。
            \item[\texttt{credits}] 首页底部的谢辞。
        \end{description}

        文稿行为:
        \begin{description}
            \item[\texttt{showtocsubsection}] 是否在目录中显示二级标题。初始值为\texttt{false}。可指定\texttt{true}或\texttt{false}。
            \item[\texttt{showsubsectionpage}] 是否在小节开始处添加标题页。初始值为\texttt{false}。可指定\texttt{true}或\texttt{false}。
        \end{description}
    
    \end{frame}

    \begin{frame}
        \frametitle{可用的选项(续)}

        标志:
        \begin{description}
            \item[\texttt{titlelogo}] 首页标志图片的路径,可以设为空来禁用。
            \item[\texttt{framelogo}] 页面头部标志图片的路径,可以设为空来禁用。
        \end{description}
    
        字体:
        \begin{description}
            \item[\texttt{font}] 指定用于正文的字体族。 选项包括\texttt{cmbr}(Cambridge Bright字体族)、 \texttt{beamer}(默认字体族)、 \texttt{harmony-os} (HarmonyOS Sans字体族)。初始值为\texttt{cmbr}。如果要使用中文,请指定\texttt{harmony-os}。
        \end{description}
    \end{frame}

    \section{首页和标题页}

    \begin{frame}[fragile]{生成首页}
        有两种方法:
        \begin{enumerate}
            \item 直接使用 \verb|\maketitle| 命令。
            \item 把 \verb|\titlepage| 命令包裹在一个\texttt{plain}页面中。
        \end{enumerate}
    \end{frame}

    \begin{frame}[fragile]{首页元素}
        首页中可以显示下面的元素:
        \begin{itemize}
            \item 标题,使用 \verb|\title| 设置。
            \item 副标题,使用 \verb|\subtitle| 设置。
            \item 作者,使用 \verb|\author| 设置。
            \item 单位,使用 \verb|\institute| 设置。
            \item 日期,使用 \verb|\date| 设置。
            \item 系列,使用 \verb|\series| 设置,在首页左上角显示。
            \item 谢辞,使用 \verb|\credits| 设置,在首页底部显示。
            \item 标志,使用 \verb|\wpreset{titlelogo = <path>}| 设置。
        \end{itemize}
    \end{frame}

    \begin{frame}
        \frametitle{标题页}
        每节起始处会自动添加一个标题页。
        
        为了使结构简单,不建议在小节开头添加标题页,这个功能默认也是关闭的。
    \end{frame}

    \wpreset{showsubsectionpage}
    \subsection{小节标题页长这模样}
    \wpreset{showsubsectionpage = false}

    \section{页面元素}

    \begin{frame}{文字高亮}
        这是普通文字。

        \structure{这是结构文字。}

        \alert{这是警示文字。}

        不要像这样\structure{滥用}\alert{颜色}!
    \end{frame}

    \begin{frame}{数学}
        我们使用 \AmS\ Euler作为数学字体。

        示例公式:
        \begin{align}
            E = mc^2\\
            {x} = \left(\frac{-b \pm \sqrt{4ac - b^2}}{2a}\right)\\
            \int_a^b f(t){\rm d}t = F(b) - F(a)
        \end{align}
    \end{frame}

    \begin{frame}[fragile]{文本块}
        文本块看起来很现代:

        \begin{block}{这是标题}
            这是内容。
        \end{block}

        \begin{block}{}
            \applyparskip
            能不能没有标题呢?
            \tcblower
            当然可以,也不会有多出来的空间!
            
            文本块使用的是 \texttt{tcolorbox},所以可以利用 \verb|\tcblower| 来分割内容。
        \end{block}
    \end{frame}

    \begin{frame}{文本块(续)}
        \begin{block}{孔乙己}
            \linespread{1.3}
            \setlength{\parindent}{2em}
            孔乙己喝过半碗酒,涨红的脸色渐渐复了原,旁人便又问道,“孔乙己,你当真认识字么?”孔乙己看着问他的人,显出不屑置辩的神气。他们便接着说道,“你怎的连半个秀才也捞不到呢?”孔乙己立刻显出颓唐不安模样,脸上笼上了一层灰色,嘴里说些话;这回可是全是之乎者也之类,一些不懂了。在这时候,众人也都哄笑起来:店内外充满了快活的空气。
        \end{block}
    \end{frame}

    \begin{frame}{其他文本块}
        \begin{exampleblock}{示例}
            这是一个例子。
        \end{exampleblock}

        \begin{alertblock}{警告}
            这是一个警告!
        \end{alertblock}
    \end{frame}

    \begin{frame}[fragile]{数学块}
        \begin{wdefinition}[math block]
            A \structure{\emph{math block}} is a block containing mathematical content.
        \end{wdefinition}

        \alert{Note:} the names of the mathematical environments provided by \texttt{wpre} are all prefixed with `\texttt{w}' in order to distinguish from the original ones.

        \begin{wexample}
            To display a definition block, write
            \begin{verbatim}
\begin{wdefinition}[optional name]
    ...
\end{wdefinition}\end{verbatim}
        \end{wexample}
    \end{frame}

    \begin{frame}{Math Blocks (cont'd)}
        \begin{wlemma}
            For $a\perp b$, we have $\Vert a + b\Vert^2 = \Vert a\Vert^2 + \Vert b\Vert^2$.
        \end{wlemma}

        \begin{wtheorem}
            This theorem is correct.
        \end{wtheorem}

        \begin{proof}
            That's obvious.
        \end{proof}
    \end{frame}

    \begin{frame}{Figures}
        \begin{figure}
            \includegraphics[height=3em]{../assets/biglogo-red.png}
            \caption{SJTU Logo}
        \end{figure}
    \end{frame}

    \section{其他}

    \begin{frame}[fragile]{Other tips}
        \begin{itemize}
            \item We only support a frame title, setting subtitles for frames has no effect.
            \item The \verb|\part| command hasn't been tested.
        \end{itemize}
    \end{frame}

    \begin{frame}
        \frametitle{参考文献示例}
        \begin{thebibliography}{Dijkstra, 1982}
        \bibitem[Salomaa, 1973]{Salomaa1973}
          A.~Salomaa.
          \newblock {Formal Languages}.
          \newblock Academic Press, 1973.
        \bibitem[Dijkstra, 1982]{Dijkstra1982}
          E.~Dijkstra.
          \newblock Smoothsort, an alternative for sorting in situ.
          \newblock {Science of Computer Programming}, 1(3):223--233, 1982.
        \end{thebibliography}
      \end{frame}

    \begin{frame}[fragile]
        \frametitle{彩蛋:用 \LaTeX 3 寻找素数吧!}
        \ExplSyntaxOn
            \int_const:Nn \c_check_bound_int {1000}
            \intarray_new:Nn \l_isprime_intarray {\c_check_bound_int}
            \seq_new:N \l_primes_seq
            \int_new:N \l_prime_cnt_int

            \int_new:N \l_current_prime_idx_int
            \int_new:N \l_current_prime_int
            
            % mark all numbers as potential primes.
            \int_step_inline:nn {\c_check_bound_int} {
                \intarray_gset:Nnn \l_isprime_intarray {#1} {
                    \int_eval:n {1}
                }
            }

            % rule out composites
            \int_step_inline:nnn {2} {\c_check_bound_int} {
                \int_set:Nn \l_tmpa_int {
                    \intarray_item:Nn \l_isprime_intarray {#1}
                }

                % update prime sequence
                \int_compare:nNnT {\l_tmpa_int} = {1} {
                    \seq_put_right:Nn \l_primes_seq {#1}
                    \int_incr:N \l_prime_cnt_int 
                }

                % ruling out
                \int_set:Nn \l_current_prime_idx_int {0}
                \bool_set_true:N \l_tmpa_bool
                \int_do_until:nn {
                    \bool_if:nTF {\l_tmpa_bool} {1} {0} = 0
                } {
                    \int_incr:N \l_current_prime_idx_int
                    \int_set:Nn \l_current_prime_int {
                        \seq_item:Nn \l_primes_seq {\l_current_prime_idx_int}
                    }

                    \int_set:Nn \l_tmpb_int {#1 * \l_current_prime_int}
                    
                    \int_compare:nNnTF {\l_tmpb_int} > {\c_check_bound_int} {
                        \bool_set_false:N \l_tmpa_bool
                    } {
                        \intarray_gset:Nnn \l_isprime_intarray {\l_tmpb_int} {0}
                    }

                    % linear time method
                    \int_compare:nT {\int_mod:nn {#1} {\l_current_prime_int} = 0} {
                        \bool_set_false:N \l_tmpa_bool
                    }

                    \int_compare:nNnT {\l_current_prime_idx_int} = {\l_prime_cnt_int} {
                        \bool_set_false:N \l_tmpa_bool
                    }
                }
            }

            \par 
            The~primes~in~the~first~\int_use:N \c_check_bound_int \ numbers~are:
            \par
            \begin{quote}
                \seq_use:Nn \l_primes_seq {,~} .
            \end{quote}
        \ExplSyntaxOff
    \end{frame}
\end{document}
