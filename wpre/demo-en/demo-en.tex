\documentclass[no-math, aspectratio=1610, 10pt]{beamer}
\usetheme{wpre}

\usepackage{amsmath}
\usepackage{amsfonts}
\usepackage{amssymb}
\usepackage{mathabx}
\usepackage{verbatim}
\usepackage{expl3}
\usepackage{lipsum}


\title{Beamer theme \texttt{wpre}}
\subtitle{A quick tour}
\series{Hello, \texttt{wpre}!}
\credits{Thanks IPADS, thier slides theme inspired this a lot.}
\author{WunschUnreif}
\institute{Basics, SJTU}
\date{\today}

\begin{document}
    \maketitle

    \begin{frame}{Contents}
        \tableofcontents
    \end{frame}

    \section{Theme Options}

    \begin{frame}[fragile]
        \frametitle{Passing options to \texttt{wpre}}
        You can pass options when loading this beamer theme in your preamble:
        \begin{verbatim}
    \usetheme[key = val, ...]{wpre}
        \end{verbatim}
        Alternatively, options can be modified in your document using the macro 
        \begin{verbatim}
    \wpreset{key = val, ...}
        \end{verbatim}
    \end{frame}

    \begin{frame}
        \frametitle{Available options}
    
        Metadata:
        \begin{description}
            \item[\texttt{series}] Series name of your presentation, course name for example.
            \item[\texttt{credits}] Credits shown at the bottom on the title page. 
        \end{description}

        Behavior control:
        \begin{description}
            \item[\texttt{showtocsubsection}] Controls whether subsections are shown in the table of contents. Initially \texttt{false}. Options are \texttt{true} and \texttt{false}.
            \item[\texttt{showsubsectionpage}] Controls whether a title page is added at the beginning of each subsection. Initially \texttt{false}. Options are \texttt{true} and \texttt{false}.
        \end{description}
    
    \end{frame}

    \begin{frame}
        \frametitle{Available options (cont'd)}

        Logo:
        \begin{description}
            \item[\texttt{titlelogo}] Path to the logo image shown on the title page. Set to empty to disable.
            \item[\texttt{framelogo}] Path to the logo image used for frame titles. Set to empty to disable. 
        \end{description}
    
        Fonts:
        \begin{description}
            \item[\texttt{font}] Font family to use for main text. Options are \texttt{cmbr} (Cambridge Bright), \texttt{beamer} (the default family), \texttt{harmony-os} (HarmonyOS Sans), and \texttt{sourcehan} (Source Han Sans). Initially \texttt{cmbr}. Choose \texttt{harmony-os} or \texttt{sourcehan} if you are writting in chinese.
        \end{description}
    \end{frame}

    \section{Title Page and Section Pages}

    \begin{frame}[fragile]{How to make the title page?}
        \framesubtitle{Use \texttt{maketitle}}
        Two ways:
        \begin{enumerate}
            \item Use \verb|\maketitle| directly. (recommended)
            \item Wrap \verb|\titlepage| inside a frame with options \texttt{plain, standout}.
            \begin{enumerate}
                \item sub item
                \begin{enumerate}
                    \item sub sub item
                \end{enumerate}
            \end{enumerate}
        \end{enumerate}

        If you don't like the standout title page, it is also feasible to put the \verb|\titlepage| in a \texttt{plain} frame. Remember to change the logo accordingly.
    \end{frame}

    \begin{frame}[fragile]{What's on the title page}
        There are several elements displayed on the title page:
        \begin{itemize}
            \item Title, set with \verb|\title|.
            \item Subtitle, set with \verb|\subtitle|.
            \item Author, set with \verb|\author|.
            \item Institute, set with \verb|\institute|.
            \item Date, set with \verb|\date|.
            \item Series, displayed in the top left corner, set with \verb|\series|.
            \item Credits, displayed at the bottom, set with \verb|\credits|.
            \item Logo, set with \verb|\wpreset{titlelogo = <path>}|.
            \begin{itemize}
                \item sub item
                \begin{itemize}
                    \item subsub item
                \end{itemize}
            \end{itemize}
        \end{itemize}
    \end{frame}

    \begin{frame}
        \frametitle{Section pages}
        At the beginning of each section, a section page is automatically generated. 
        
        We don't recommend the use of subsections since the structure of your presentation should be concise and flat. So \textbf{no} subsection page is provided by default. But you can change this behavior by setting relevant options.
    \end{frame}

    \wpreset{showsubsectionpage}
    \subsection{Subsection page looks like this!}
    \wpreset{showsubsectionpage = false}

    \section{Inside Frames}

    \begin{frame}{Highlighting}
        Normal text looks like this.

        \structure{Structure text looks like this.}

        \alert{Alerted text looks like this.}

        \structure{Colors} should be used \alert{parsimoniously}, not like this.
    \end{frame}

    \begin{frame}{Mathematics}
        The theme uses the \AmS\ Euler font designed by Zapf.

        Formulas:
        \begin{align}
            E = mc^2\\
            {x} = \left(\frac{-b \pm \sqrt{4ac - b^2}}{2a}\right)\\
            \int_a^b f(t){\rm d}t = F(b) - F(a)
        \end{align}
    \end{frame}

    \begin{frame}[fragile]{Blocks}
        The blocks looks fancy!

        \begin{block}{This is a block}
            The block has this sentence as its content.
        \end{block}

        \begin{block}{}
            What if a block has no title?
            \tcblower
            As you wish, no extra space! 
            
            By the way, blocks are actually \texttt{tcolorbox}es, where you can use \verb|\tcblower| to separate the contents.
        \end{block}
    \end{frame}

    \begin{frame}{Blocks (cont'd)}
        \begin{block}{Some long text}
            \lipsum[3]
        \end{block}
    \end{frame}

    \begin{frame}{Special blocks}
        \begin{exampleblock}{Example blocks}
            Example looks green.
        \end{exampleblock}

        \begin{alertblock}{Alert blocks}
            This is a warning!
        \end{alertblock}
    \end{frame}

    \begin{frame}[fragile]{Math blocks}
        \begin{wdefinition}[math block]
            A \structure{\emph{math block}} is a block containing mathematical content.
        \end{wdefinition}

        \alert{Note:} the names of the mathematical environments provided by \texttt{wpre} are all prefixed with `\texttt{w}' in order to distinguish from the original ones.

        \begin{wexample}
            To display a definition block, write
            \begin{verbatim}
\begin{wdefinition}[optional name]
    ...
\end{wdefinition}\end{verbatim}
        \end{wexample}
    \end{frame}

    \begin{frame}{Math Blocks (cont'd)}
        \begin{wlemma}
            For $a\perp b$, we have $\Vert a + b\Vert^2 = \Vert a\Vert^2 + \Vert b\Vert^2$.
        \end{wlemma}

        \begin{wtheorem}
            This theorem is correct.
        \end{wtheorem}

        \begin{proof}
            That's obvious.
        \end{proof}
    \end{frame}

    \begin{frame}{Figures}
        \begin{figure}
            \includegraphics[height=3em]{../assets/biglogo-red.png}
            \caption{SJTU Logo}
        \end{figure}
    \end{frame}

    \section{Other Tips}

    \begin{frame}[fragile]{Other tips}
        \begin{itemize}
            \item We only support a frame title, setting subtitles for frames has no effect.
            \item The \verb|\part| command hasn't been tested.
        \end{itemize}
    \end{frame}

    \begin{frame}
        \frametitle{Bibliography}
        \begin{thebibliography}{Dijkstra, 1982}
        \bibitem[Salomaa, 1973]{Salomaa1973}
          A.~Salomaa.
          \newblock {Formal Languages}.
          \newblock Academic Press, 1973.
        \bibitem[Dijkstra, 1982]{Dijkstra1982}
          E.~Dijkstra.
          \newblock Smoothsort, an alternative for sorting in situ.
          \newblock {Science of Computer Programming}, 1(3):223--233, 1982.
        \end{thebibliography}
      \end{frame}

    \begin{frame}[fragile]
        \frametitle{Finding Primes Using \LaTeX 3!}
        \ExplSyntaxOn
            \int_const:Nn \c_check_bound_int {1000}
            \intarray_new:Nn \l_isprime_intarray {\c_check_bound_int}
            \seq_new:N \l_primes_seq
            \int_new:N \l_prime_cnt_int

            \int_new:N \l_current_prime_idx_int
            \int_new:N \l_current_prime_int
            
            % mark all numbers as potential primes.
            \int_step_inline:nn {\c_check_bound_int} {
                \intarray_gset:Nnn \l_isprime_intarray {#1} {
                    \int_eval:n {1}
                }
            }

            % rule out composites
            \int_step_inline:nnn {2} {\c_check_bound_int} {
                \int_set:Nn \l_tmpa_int {
                    \intarray_item:Nn \l_isprime_intarray {#1}
                }

                % update prime sequence
                \int_compare:nNnT {\l_tmpa_int} = {1} {
                    \seq_put_right:Nn \l_primes_seq {#1}
                    \int_incr:N \l_prime_cnt_int 
                }

                % ruling out
                \int_set:Nn \l_current_prime_idx_int {0}
                \bool_set_true:N \l_tmpa_bool
                \int_do_until:nn {
                    \bool_if:nTF {\l_tmpa_bool} {1} {0} = 0
                } {
                    \int_incr:N \l_current_prime_idx_int
                    \int_set:Nn \l_current_prime_int {
                        \seq_item:Nn \l_primes_seq {\l_current_prime_idx_int}
                    }

                    \int_set:Nn \l_tmpb_int {#1 * \l_current_prime_int}
                    
                    \int_compare:nNnTF {\l_tmpb_int} > {\c_check_bound_int} {
                        \bool_set_false:N \l_tmpa_bool
                    } {
                        \intarray_gset:Nnn \l_isprime_intarray {\l_tmpb_int} {0}
                    }

                    % linear time method
                    \int_compare:nT {\int_mod:nn {#1} {\l_current_prime_int} = 0} {
                        \bool_set_false:N \l_tmpa_bool
                    }

                    \int_compare:nNnT {\l_current_prime_idx_int} = {\l_prime_cnt_int} {
                        \bool_set_false:N \l_tmpa_bool
                    }
                }
            }

            \par 
            The~primes~in~the~first~\int_use:N \c_check_bound_int \ numbers~are:
            \par
            \begin{quote}
                \seq_use:Nn \l_primes_seq {,~} .
            \end{quote}
        \ExplSyntaxOff
    \end{frame}
\end{document}
