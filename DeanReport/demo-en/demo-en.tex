\documentclass[showbib, lang=en]{DeanReport}

\usepackage{lipsum}
\usepackage{xltxtra}
\usepackage{longtable}

\title{Weekly Report}
\author{Dean Liu}
\deaninfo{
    author      = WunschUnreif,
    start       = 6/5,
    end         = 6/11,
    emotion     = 3,
    progress    = 6,
    positivity  = 9,
}

\begin{document}

\maketitle

\section{Usage}
The following line of code loads all functionalities of this document class.

\begin{codeblock*}
\begin{minted}[autogobble]{latex}
    \documentclass{DeanReport}
\end{minted}
\end{codeblock*}

The compilation requires \XeLaTeX. If you also want to display code blocks using the package \code{minted}, the following command is recommended.

\begin{codeblock*}
\begin{minted}[autogobble]{bash}
    xelatex -shell-escape <your tex root file.tex>
\end{minted}
\end{codeblock*}

For generating a bibliography, you need to run the recipe \code{xe->bib->xe->xe}.


\section{Class Options}
You can pass options to this class by \code{\textbackslash documentclass[<kvpairs>]\{DeanReport\}}. Here lists all the options defined by DeanReport. Extra arguments will be passed to the base class \code{article} or \code{ctexart} depending on the language choice.

\begin{table}[!h]
    \centering
    \renewcommand{\arraystretch}{1.2}
    \begin{tabular}{lp{30mm}llp{50mm}}
        \hline
        Option Name & Choices & Default & Initial & Explanation\\
        \hline
        \code{lang} & \code{en}, \code{zh} & & \code{en} & Choose the primary language\\
        \code{showlineno} & \code{on}, \code{off} &\code{on} & \code{off} & Switch the line numbers\\
        \code{showbib} & \code{on}, \code{off} &\code{on} & \code{off} & Switch the bibliography\\
        \code{bibstyle} &  &  & \code{abbrv} & Set the bibliography style\\
        \hline
    \end{tabular}
\end{table}

For example, we may want to set Chinese as the primary language, turn on line numbers and bibliography. Also, we want to set \code{alpha} as the bibliography style. To achieve this setting, we write the following code.

\begin{codeblock}[title = An example of passing options]
\begin{minted}[autogobble]{latex}
    \documentclass[
        lang        = zh,
        showlineno,
        showbib,
        bibstyle    = alpha
    ]{DeanReport}
\end{minted}
\end{codeblock}

\paragraph{Difference between `default' and `initial'} An option may have a default value together with an initial value. The initial value is used if you had never passed the option. On the other hand, the default value is used when you pass the option only with its key but don't specify a value. For example,
\begin{codeblock*}
\begin{minted}[autogobble]{latex}
    \documentclass[showbib]{DeanReport}
\end{minted}
\end{codeblock*}
is equivalent to 
\begin{codeblock*}
\begin{minted}[autogobble]{latex}
    \documentclass[showbib=on]{DeanReport}
\end{minted}
\end{codeblock*}
since the default value of \code{showbib} is \code{on}.
\par 
If \code{showbib} isn't passed, e.g. \code{\textbackslash documentclass\{DeanReport\}}, then the initial value \code{off} will be used by default.


\section{Commands and Environments}
\subsection{Command \texorpdfstring{\code{\textbackslash deanset}}{deanset} }

You can change any settings of this document class using \code{\textbackslash deanset}. The settings, including class options, are organized as a hierarchical key-value database. You may consult the source code of this class to understand each setting. As an example, you can change the title and the default color of code blocks by the following code.
\begin{codeblock*}
\begin{minted}[autogobble]{latex}
    \deanset{
        codeblocksettings = {
            color = \accentcolorname,
            caption = Another Listing Title
        }
    }
\end{minted}
\end{codeblock*}

\begin{deanbox}[\texttt{\textbackslash deanset} ignores groups][\warningcolorname]
    It is important to remember that the effect of \code{\textbackslash deanset} is global. No matter where you use this command, its effect will preserve until you call this command again.
\end{deanbox}

\subsection{Command \code{\textbackslash deaninfo}}
You can set the basic information of your report using \code{\textbackslash deaninfo}. It takes one argument which is a list of key-value pairs. The entries you can set are listed below.
\begin{center}
    \renewcommand{\arraystretch}{1.2}
    \begin{longtable}{llllp{50mm}}
        \hline
        Entry Name & Choices & Initial & Explanation\\
        \hline
        \code{title} &  & Weekly Report & The title of your report\\
        \code{author} &  & Hongzhou Liu & Your name\\\hline
        \\
        \code{start} &  & [Start Date Here] & Start date of the week\\
        \code{end} &  & [End Date Here] & End date of the week\\
        \code{emotion} & 0, $\ldots$, 9 & 7 & Emotion index\\
        \code{progress} & 0, $\ldots$, 9 & 7 & Progress index\\
        \code{positivity} & 0, $\ldots$, 9 & 7 & Positivity index\\
        \code{bib} & & \texttt{references} & Bib file name\\
        \hline
    \end{longtable}
\end{center}

The information setting of this document is
\begin{codeblock*}
\begin{minted}[autogobble]{latex}
    \deaninfo{
        author      = WunschUnreif,
        start       = Jul. 5,
        end         = Jul. 11,
        emotion     = 3,
        progress    = 6,
        positivity  = 9,
    }
\end{minted}
\end{codeblock*}

\subsection{Box Generating}
We provide a fantastic box environment and inline boxes in this class. Detailed description can be found in section \ref{sec:boxes}.

\subsection{Environments \code{codeblock} and \code{codeblock*}}
Code presentation is a ubiquitous need. We provide environment \code{codeblock} and its starred version \code{codeblock*} for you. The only difference is that the unstarred version maintains a counter for numbering code blocks while \code{codeblock*} doesn't provide numbering.
They take an optional key-value parameter where you can specify the title and the color with key \code{title} and \code{color} respectively.
\par 
Note however these environments only provides a box. Inside them you need to use other environments to typeset the code itself. We highly recommend the \code{minted} package for displaying code.
\par 
As an example, the following code 
\begin{codeblock*}
\inputminted[autogobble]{latex}{examplecode}
\end{codeblock*}
yields the block
\begin{codeblock}[title = Hello World in Python, color = pink]
\begin{minted}[autogobble]{python}
    print("Hello, World!")
\end{minted}
\end{codeblock}

\subsection{Theorem-like Environments}
We provide environments \code{theorem}, \code{lemma}, \code{corollary}, \code{definition}, \code{example}, \code{remark} just like the standard \code{amsthm} environments, but implemented independently to support a colorful box. These environments share a common counter which is reset at each section. Each of these environments has a corresponding starred version which doesn't step the counters. As before, these environments take an optional argument which is their title. A demo of these environments are given in section \ref{sec:math}.


\section{Math}
\label{sec:math}

\subsection{Formulas}
An example:
\begin{displaymath}
    \mathbf{B}(P) = \frac{\mu_0}{4\pi}\int\frac{\mathbf{I}\times\hat{r}'}{r'^2}{\rm d}l = \frac{\mu_0}{4\pi} I \int \frac{{\rm d}\bm{l}\times \hat{r}'}{r'^2}
\end{displaymath}

Note that the \code{displaymath} environment (equivalently \code{\textbackslash[...\textbackslash]}) is preferred to the old \code{\$\$}.

\subsection{Math Fonts}
There are several math fonts you can use:
\allowdisplaybreaks
\begin{align}
    ABCDEFGHIJKLMNOPQRSTUVWXYZ\\
    abcdefghijklmnopqrstuvwxyz\\
    \mathbf{abcdefghijklmnopqrstuvwxyz}\\
    \bm{abcdefghijklmnopqrstuvwxyz}\\
    \mathbb{ABCDEFGHIJKLMNOPQRSTUVWXYZ}\\
    \mathcal{ABCDEFGHIJKLMNOPQRSTUVWXYZ}\\
    \mathscr{ABCDEFGHIJKLMNOPQRSTUVWXYZ}\\
    \mathfrak{ABCDEFGHIJKLMNOPQRSTUVWXYZ}
\end{align}

\subsection{Theorems}

\begin{theorem}[named theorem]
    This is a theorem with name.
\end{theorem}

\begin{theorem}
    No name theorem.
\end{theorem}

\begin{lemma}
    This is a lemma.
\end{lemma}

\begin{corollary}
    This is a corollary.
\end{corollary}

\begin{definition}
    This is a definition.
\end{definition}

\begin{example}
    This is an example.
\end{example}

\begin{remark}
    This is a remark.
\end{remark}

\begin{remark*}
    Each theorem-like environment has a starred version which doesn't increase nor show the counter.
\end{remark*}

\begin{proof}
    This document class is a trivial result.
\end{proof}

\section{Boxes}
\label{sec:boxes}
\begin{deanbox}[A fantastic box]
    You can use the box with environment \verb|deanbox|.
\end{deanbox}

\begin{deanbox}[]
    A \verb|deanbox| takes 2 optional arguments. The first one is the title of the box. Of course you can get a box without title.
\end{deanbox}

\begin{deanbox}[Setting box colors][\successcolorname]
    The second optional argument controls the color of the box. You should only pass the color series name such as \verb|green| or use the theme color such as \verb|\successcolorname|. By default, the box frame and the title bar use the base color and the box background uses the \verb|-lighten-5| variation.
\end{deanbox}

There is also an inline version of boxes using command \code{\textbackslash inlbox}. Since inline boxes are preferable to display short codes, we also provide the command \code{\textbackslash code} for such purposes, which is equivalent to a \code{\textbackslash texttt} inside a \code{\textbackslash inlbox}.


\section{Enumerate Environments}
An \code{enumerate}:
\begin{enumerate}
    \item enum 1
    \item enum 2
    \item enum paragraph \lipsum[2]
\end{enumerate}
Large numbers:
\begin{enumerate}[start=9]
    \item enum 1
    \item enum 2
    \item enum paragraph \lipsum[2]
\end{enumerate}

A nested \code{enumerate}:
\begin{enumerate}
    \item First item.
    \item Nested 
    \begin{enumerate}
        \item Item 2.1
        \item Item 2.2
        \item Level 3:
        \begin{enumerate}
            \item Item 2.3.1
            \item Item 2.3.2
        \end{enumerate}
    \end{enumerate}
\end{enumerate}

An \code{description}:
\begin{description}
    \item[First] first item.
    \item[Second] second item.
\end{description}

An \code{itemize}:
\begin{itemize}
    \item First item.
    \item Nested 
    \begin{itemize}
        \item Item 2.1
        \item Item 2.2
        \item Level 3:
        \begin{itemize}
            \item Item 2.3.1
            \item Item 2.3.2
        \end{itemize}
    \end{itemize}
\end{itemize}


\section{Lorem Ipsum}
\lipsum

\section{Acknowledgements}
    We thank the document \code{clsguide.pdf}\cite{clsguide} for a good instruction on how to write \LaTeX{} classes.

    We thank the packages \code{kvoptions}, \code{kvdefinekeys} and \code{kvsetkeys} for providing an easy-to-use key-value method for option managements.

    We also thank the package \code{tcolorbox} for providing a powerful package to draw beautiful boxes.

    Finally, we thank the package \code{minted} together with the engine \code{Pygmentize} as they provided a modern typeset for codes.

\end{document}
